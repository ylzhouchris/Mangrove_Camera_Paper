\documentclass{article}


\usepackage{arxiv}

\usepackage[utf8]{inputenc} % allow utf-8 input
\usepackage[T1]{fontenc}    % use 8-bit T1 fonts
\usepackage{hyperref}       % hyperlinks
\usepackage{url}            % simple URL typesetting
\usepackage{booktabs}       % professional-quality tables
\usepackage{amsfonts}       % blackboard math symbols
\usepackage{nicefrac}       % compact symbols for 1/2, etc.
\usepackage{microtype}      % microtypography
\usepackage{lipsum}

\title{Emonitoring the Growth of Mangroves using Camera Images}


\author{
  Q. Xiang\thanks{Use footnote for providing further
    information about author (webpage, alternative
    address)---\emph{not} for acknowledging funding agencies.} \\
  Department of Computer Science\\
  Cranberry-Lemon University\\
  Pittsburgh, PA 15213 \\
  \texttt{hippo@cs.cranberry-lemon.edu} \\
  %% examples of more authors
   \And
 Elias D.~Striatum \\
  Department of Electrical Engineering\\
  Mount-Sheikh University\\
  Santa Narimana, Levand \\
  \texttt{stariate@ee.mount-sheikh.edu} \\
  %% \AND
  %% Coauthor \\
  %% Affiliation \\
  %% Address \\
  %% \texttt{email} \\
  %% \And
  %% Coauthor \\
  %% Affiliation \\
  %% Address \\
  %% \texttt{email} \\
  %% \And
  %% Coauthor \\
  %% Affiliation \\
  %% Address \\
  %% \texttt{email} \\
}

\begin{document}
\maketitle

\begin{abstract}
\lipsum[1]
\end{abstract}


% keywords can be removed
\keywords{First keyword \and Second keyword \and More}


\section{Introduction}
\lipsum[2]
\lipsum[3]


\section{Methods and Materials}
\label{sec:headings}


\subsection{Data}
On the mangrove forest observation tower, a digital camera was installed in a fixed position, leveled with the horizon, with a view across the top of the canopy. This TLCAM PRO camera provided JPG images at 30-min intervals for 24h per day. Each image has three color channels of 8-bit RGB color information and for each channel digital numbers (DN) ranging from 0 to 255. Cameras were operated in the automatic aperture and exposure mode. The timestamps of all images were extracted from the Exif information stored along with the images. The Exif data stores the basic information about the image, including image gamut space, capture time and save time.We save all the capture times to a timestamp dictionary for fast query. In total, we captured 15,629 images in the period from xxxx to xxx.  Figure 1 shows two image examples and the extracted timestamps.

++ Figure 1 
++ Study Area


\begin{equation}
\xi _{ij}(t)=P(x_{t}=i,x_{t+1}=j|y,v,w;\theta)= {\frac {\alpha _{i}(t)a^{w_t}_{ij}\beta _{j}(t+1)b^{v_{t+1}}_{j}(y_{t+1})}{\sum _{i=1}^{N} \sum _{j=1}^{N} \alpha _{i}(t)a^{w_t}_{ij}\beta _{j}(t+1)b^{v_{t+1}}_{j}(y_{t+1})}}
\end{equation}

\subsubsection{Data Quality Control}
++ Quality control before and during implementation. Keep the FOV constant. Automatic aperture and exposure mode. 
Time series were first visually inspected for camera shifts and the related changes in field of view (FOV). The FOV remained largely unchanged, however, some larger shifts in the FOV were observed as the positioning of the camera changed a few times throughout the study. We also observed images with unusual illumination which may affect the calculation of vegetation indices. For example, images before sun rise and after sun set had inadequate illumination and so can hardly be used for monitoring the greenness. Also, images around the sunrise and sunset had significant different hues. 
To wipe out the bad images significantly affected by camera shifting or unusual illumination, we calculated the percentage of green color in each image as an indicator for filtering and calibration. The percentage of green color was computed as the percentage of image pixels where the DN(g) was smaller than 175 (see Figure S1 for more details++ histogram in Supporting Information). After applying the greenness proportion calculation to one month’s images, we found that the proportions are segmented to two parts. Irregular images suffering from bad illumination or camera shifting problems had values being extremely close to 1, while the normal ones were normally distributed around 0.6. Therefore, we selected a benchmark of 0.75 to sift out the normal images to be used in further calculations (++ histogram in Supporting Information). Using this filter, we identified and removed XXX images  from the original dataset. Rechecking the identified bad images, we found that they are all ill-illuminated and had a time span from 6pm to the second day’s 6am.
Moreover, at the bottom of each image, a black bar is attached to display the time, temperature and equipment manufactory, which largely disturbed our calculation of vegetation indices. Noting these noises, images were first cropped to exclude the bars. The way to solve the field of view shifts is discussed in later sections.
All image processing were conducted using Python 3.6. 




\paragraph{Paragraph}


\subsection{Calculation of Vegetation Indices}
To quantify canopy greenness, we calculated the green chromatic coordinate (GCC), which is widely used to monitor canopy development and is strongly related to GPP (xxxx cite), can be denoted as follows,
GCC=DN(g)/(DN(r)+DN(g)+DN(b))

where the subscripts R, G, and B indicate the red, green, and blue channels, respectively. 
We also computed another vegetation index, the excess green (EXGCC) index, which can be denoted as follows, 
EXGCC=2DN(g)-(DN(r)+DN(b)) 

EXGCC can be less noisy than GCC in some ecosystems, notably in conifer canopies (xxx cite), although GCC is generally more effective than EXGCC in suppressing the effects of changes in scene illumination (xxx cite). Due to the formulas, all the GCCs which are larger than 1 is dropped. All the EXGCCs which are greater than 200 are modified to 200. 


\subsection{Time-Series Smoothing}
Knowing the vegetation indices are sensitive to illumination conditions, we extracted the images ranging from 10am to 2pm during the day to avoid hue changes due to sunset and to have the most stable illumination. We applied a moving window smoothing to the extracted VIs to highlight the underlying monthly and seasonal change patterns by removing several extremities and unexpected fluctuations that appeared in the extracted time series of GCC and EXGCC caused by factors such as slight illumination disturbance, droplets on lens, and cloud. Several smoothing methods including the mean filter, median filter, and 90\% percentile filter were applied for comparison. Following the practice of xxxx (cite), the window size was selected as 3 days. 


\subsection{Validation}





\section{Examples of citations, figures, tables, references}
\label{sec:others}
\lipsum[8] \cite{kour2014real,kour2014fast} and see \cite{hadash2018estimate}.

The documentation for \verb+natbib+ may be found at
\begin{center}
  \url{http://mirrors.ctan.org/macros/latex/contrib/natbib/natnotes.pdf}
\end{center}
Of note is the command \verb+\citet+, which produces citations
appropriate for use in inline text.  For example,
\begin{verbatim}
   \citet{hasselmo} investigated\dots
\end{verbatim}
produces
\begin{quote}
  Hasselmo, et al.\ (1995) investigated\dots
\end{quote}

\begin{center}
  \url{https://www.ctan.org/pkg/booktabs}
\end{center}


\subsection{Figures}
\lipsum[10] 
See Figure \ref{fig:fig1}. Here is how you add footnotes. \footnote{Sample of the first footnote.}
\lipsum[11] 

\begin{figure}
  \centering
  \fbox{\rule[-.5cm]{4cm}{4cm} \rule[-.5cm]{4cm}{0cm}}
  \caption{Sample figure caption.}
  \label{fig:fig1}
\end{figure}

\subsection{Tables}
\lipsum[12]
See awesome Table~\ref{tab:table}.

\begin{table}
 \caption{Sample table title}
  \centering
  \begin{tabular}{lll}
    \toprule
    \multicolumn{2}{c}{Part}                   \\
    \cmidrule(r){1-2}
    Name     & Description     & Size ($\mu$m) \\
    \midrule
    Dendrite & Input terminal  & $\sim$100     \\
    Axon     & Output terminal & $\sim$10      \\
    Soma     & Cell body       & up to $10^6$  \\
    \bottomrule
  \end{tabular}
  \label{tab:table}
\end{table}

\subsection{Lists}
\begin{itemize}
\item Lorem ipsum dolor sit amet
\item consectetur adipiscing elit. 
\item Aliquam dignissim blandit est, in dictum tortor gravida eget. In ac rutrum magna.
\end{itemize}


\bibliographystyle{unsrt}  
%\bibliography{references}  %%% Remove comment to use the external .bib file (using bibtex).
%%% and comment out the ``thebibliography'' section.


%%% Comment out this section when you \bibliography{references} is enabled.
\begin{thebibliography}{1}

\bibitem{kour2014real}
George Kour and Raid Saabne.
\newblock Real-time segmentation of on-line handwritten arabic script.
\newblock In {\em Frontiers in Handwriting Recognition (ICFHR), 2014 14th
  International Conference on}, pages 417--422. IEEE, 2014.

\bibitem{kour2014fast}
George Kour and Raid Saabne.
\newblock Fast classification of handwritten on-line arabic characters.
\newblock In {\em Soft Computing and Pattern Recognition (SoCPaR), 2014 6th
  International Conference of}, pages 312--318. IEEE, 2014.

\bibitem{hadash2018estimate}
Guy Hadash, Einat Kermany, Boaz Carmeli, Ofer Lavi, George Kour, and Alon
  Jacovi.
\newblock Estimate and replace: A novel approach to integrating deep neural
  networks with existing applications.
\newblock {\em arXiv preprint arXiv:1804.09028}, 2018.

\end{thebibliography}


\end{document}
